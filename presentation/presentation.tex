%%%%%%%%%%%%%%%%%%%%%%%%%%%%%%%%%%%%%%%%%
% Beamer Presentation
% LaTeX Template
% Version 1.0 (10/11/12)
%
% This template has been downloaded from:
% http://www.LaTeXTemplates.com
%
% License:
% CC BY-NC-SA 3.0 (http://creativecommons.org/licenses/by-nc-sa/3.0/)
%
%%%%%%%%%%%%%%%%%%%%%%%%%%%%%%%%%%%%%%%%%

%----------------------------------------------------------------------------------------
%	PACKAGES AND THEMES

%----------------------------------------------------------------------------------------

\documentclass{beamer}

\usepackage[utf8]{inputenc}
\usepackage{alphabeta}
\usepackage{circuitikz}


\mode<presentation> {

% The Beamer class comes with a number of default slide themes
% which change the colors and layouts of slides. Below this is a list
% of all the themes, uncomment each in turn to see what they look like.

%\usetheme{default}
%\usetheme{AnnArbor}
%\usetheme{Antibes}
%\usetheme{Bergen}
%\usetheme{Berkeley}
%\usetheme{Berlin}
%\usetheme{Boadilla}
%\usetheme{CambridgeUS}
%\usetheme{Copenhagen}
%\usetheme{Darmstadt}
%\usetheme{Dresden}
%\usetheme{Frankfurt}
%\usetheme{Goettingen}
%\usetheme{Hannover}
%\usetheme{Ilmenau}
%\usetheme{JuanLesPins}
%\usetheme{Luebeck}
\usetheme{Madrid}
%\usetheme{Malmoe}
%\usetheme{Marburg}
%\usetheme{Montpellier}
%\usetheme{PaloAlto}
%\usetheme{Pittsburgh}
%\usetheme{Rochester}
%\usetheme{Singapore}
%\usetheme{Szeged}
%\usetheme{Warsaw}

% As well as themes, the Beamer class has a number of color themes
% for any slide theme. Uncomment each of these in turn to see how it
% changes the colors of your current slide theme.

%\usecolortheme{albatross}
%\usecolortheme{beaver}
%\usecolortheme{beetle}
%\usecolortheme{crane}
%\usecolortheme{dolphin}
%\usecolortheme{dove}
%\usecolortheme{fly}
%\usecolortheme{lily}
%\usecolortheme{orchid}
%\usecolortheme{rose}
%\usecolortheme{seagull}
%\usecolortheme{seahorse}
%\usecolortheme{whale}
%\usecolortheme{wolverine}

%\setbeamertemplate{footline} % To remove the footer line in all slides uncomment this line
%\setbeamertemplate{footline}[page number] % To replace the footer line in all slides with a simple slide count uncomment this line

%\setbeamertemplate{navigation symbols}{} % To remove the navigation symbols from the bottom of all slides uncomment this line
}

\usepackage{graphicx} % Allows including images
\usepackage{booktabs} % Allows the use of \toprule, \midrule and \bottomrule in tables

%----------------------------------------------------------------------------------------
%	TITLE PAGE
%----------------------------------------------------------------------------------------

\title[Formal SNARKs Project]{Formal SNARKs Project} % The short title appears at the bottom of every slide, the full title is only on the title page

\author{Bolton Bailey, Andrew Miller} % Your name
\institute[UIUC] % Your institution as it will appear on the bottom of every slide, may be shorthand to save space
{
University of Illinois Urbana-Champaign \\ % Your institution for the title page
\medskip
\textit{boltonb2@illinois.edu} % Your email address
}
\date{\today} % Date, can be changed to a custom date

\begin{document}

\begin{frame}
\titlepage % Print the title page as the first slide
\end{frame}

\begin{frame}
    \frametitle{Formalizing SNARKs in the Lean Theorem Prover}

    \begin{itemize}
        \item Lean seems to be an active language right now
        \item The mathlib library has support for multivariable polynomials which is critical
        \item The mathlib community is open to accepting pull requests, which has been convenient
    \end{itemize}    

\end{frame}

\begin{frame}
    \frametitle{Goals: A Library of SNARKs with Formal Proofs}
    
    We want to make it easy and convenient to formalize new SNARK proofs

    To this end, we have written tactics for facilitating proofs about polynomials and their coefficients.
    
    We also have general framework for pairing based SNARK schemes.

\end{frame}

\begin{frame}
    \frametitle{Two SNARKs formalized and more on the way}
    We have implemented a proof of the knowledge-soundness of BabySNARK and the Type III Groth '16 described in Baghery et al. in the Algebraic Group model.

    We are in the process of formalizing statements of other pairing-based SNARKs in the AGM:

    \begin{itemize}
        \item The original Groth '16 (amenable to all types of pairings)
        \item GGPR
        \item Pinocchio
    \end{itemize}    

\end{frame}

\begin{frame}
    \frametitle{The Algebraic Group Model for Linear SNARKs}

    \begin{itemize}
        \item In the AGM model we assume the proof elements are some linear combination of the CRS elements, and that the prover knows the coefficients of this linear combination. 
        \item To prove knowledge soundness, we must extract from these coefficients a satisfying witness assignment.
        \item To do this, we assume the equations that the verifier checks hold. We then prove a series of facts about the coefficients, culminating in the fact that a particular set of the coefficients satisfy the Square Span Program.
        \item From this, it follows that any algebraic adversary cannot produce a valid proof without knowing a satisfying witness.
    \end{itemize}

\end{frame}

\begin{frame}
    \frametitle{An Extremely Simple (Incomplete) Toy SNARK}

    Let $\textcolor{red}{A},\textcolor{red}{B},\textcolor{red}{C},\textcolor{red}{D},\textcolor{red}{E}$ be 0/1 valued. Make a SNARK that proves 
    $$(\textcolor{red}{A}  \textcolor{red}{D}) = \textcolor{red}{E}  \text{ or } (\textcolor{red}{B} \textcolor{red}{C}) = \textcolor{red}{E}$$

    \begin{center}
        \begin{circuitikz} 
            \node (D) at (-2,2-0.3) {$\textcolor{red}{D}$};
            \node (C) at (-2, -0.3) {$\textcolor{red}{C}$};
            \node (B) at (-2, +0.3) {$\textcolor{red}{B}$};
            \node (A) at (-2,2+0.3) {$\textcolor{red}{A}$};
            \node (E) at (3,1) {$\textcolor{red}{E}$};
            \draw
            (0,2) node[and port] (myand1) {}
            (0,0) node[and port] (myand2) {}
            (2,1) node[or port] (myor) {}
            (A) -- (myand1.in 1)
            (D) -- (myand1.in 2)
            (B) -- (myand2.in 1)
            (C) -- (myand2.in 2)
            (myor.out) -- (E)
            (myand1.out) -- (myor.in 1)
            (myand2.out) -- (myor.in 2);
        \end{circuitikz}        
    \end{center}

    Two unknown field elements \texttt{\textcolor{blue}{\alpha}, \textcolor{blue}{\beta}}
    Three CRS group elements \texttt{\textcolor{blue}{\alpha}, \textcolor{blue}{\beta}, \textcolor{blue}{\alpha\beta}}.

    Prover returns (\texttt{\textcolor{red}{A}\textcolor{blue}{\alpha}+\textcolor{red}B\textcolor{blue}{β}, \textcolor{red}C\textcolor{blue}{α}+\textcolor{red}D\textcolor{blue}{β}, \textcolor{red}E\textcolor{blue}{α}\textcolor{blue}{β}})     

    $ \implies $
    Verifier checks \texttt{(\textcolor{red}{A}\textcolor{blue}{\alpha}+\textcolor{red}B\textcolor{blue}{β})(\textcolor{red}C\textcolor{blue}{α}+\textcolor{red}D\textcolor{blue}{β}) = \textcolor{red}E\textcolor{blue}{α}\textcolor{blue}{β}}  

\end{frame}

\begin{frame}
    \frametitle{Step 0: Load Assumptions and Goal into Lean}

    Assumption:
    $$\texttt{(\textcolor{red}{A} * \textcolor{blue}{\alpha} + \textcolor{red}B * \textcolor{blue}{β})(\textcolor{red}C * \textcolor{blue}{α} + \textcolor{red}D * \textcolor{blue}{β}) = \textcolor{red}E * \textcolor{blue}{α} * \textcolor{blue}{β}}  $$   

    Goal:
    $$\texttt{(\textcolor{red}{A} * \textcolor{red}{D}) = \textcolor{red}{E}  \text{ or } (\textcolor{red}{B} * \textcolor{red}{C}) = \textcolor{red}{E}}$$

\end{frame}

\begin{frame}
    \frametitle{Step 1: Isolate Coefficients}

    Old Assumption:
    $$\texttt{(\textcolor{red}{A} * \textcolor{blue}{\alpha} + \textcolor{red}B * \textcolor{blue}{β})(\textcolor{red}C * \textcolor{blue}{α} + \textcolor{red}D * \textcolor{blue}{β}) = \textcolor{red}E * \textcolor{blue}{α} * \textcolor{blue}{β}}  $$   

    New Assumption:
    $$\texttt{\textcolor{red}{A}*\textcolor{red}{C} * \textcolor{blue}{\alpha}$^2$ + (\textcolor{red}{A}*\textcolor{red}D + \textcolor{red}{B}*\textcolor{red}C ) * \textcolor{blue}{α} * \textcolor{blue}{β} + (\textcolor{red}B*\textcolor{red}D * \textcolor{blue}{β}$^2$) = \textcolor{red}E * \textcolor{blue}{α} * \textcolor{blue}{β}}  $$   

    Goal:
    $$\texttt{(\textcolor{red}{A} * \textcolor{red}{D}) = \textcolor{red}{E}  \text{ or } (\textcolor{red}{B} * \textcolor{red}{C}) = \textcolor{red}{E}}$$

\end{frame}

\begin{frame}
    \frametitle{Step 2: Reexpress Polynomial as Equation of Coefficients}

    Old Assumption:
    $$\texttt{\textcolor{red}{A}*\textcolor{red}{C} * \textcolor{blue}{\alpha}$^2$ + (\textcolor{red}{A}*\textcolor{red}D + \textcolor{red}{B}*\textcolor{red}C ) * \textcolor{blue}{α} * \textcolor{blue}{β} + (\textcolor{red}B*\textcolor{red}D * \textcolor{blue}{β}$^2$) = \textcolor{red}E * \textcolor{blue}{α} * \textcolor{blue}{β}}  $$    

    New Assumptions:
    $$ \texttt{\textcolor{red}{A}*\textcolor{red}{C} = 0} $$
    $$ \texttt{\textcolor{red}{A}*\textcolor{red}D + \textcolor{red}{B}*\textcolor{red}C = \textcolor{red}E} $$
    $$ \texttt{\textcolor{red}B*\textcolor{red}D = 0}  $$   

    Goal:
    $$\texttt{(\textcolor{red}{A} * \textcolor{red}{D}) = \textcolor{red}{E}  \text{ or } (\textcolor{red}{B} * \textcolor{red}{C}) = \textcolor{red}{E}}$$

\end{frame}

\begin{frame}
    \frametitle{Step 3: Recursively Case on Zero Divisors and Simplify}

    Old Assumptions:
    $$ \texttt{\textcolor{red}{A}*\textcolor{red}{C} = 0} $$
    $$ \texttt{\textcolor{red}{A}*\textcolor{red}D + \textcolor{red}{B}*\textcolor{red}C = \textcolor{red}E} $$
    $$ \texttt{\textcolor{red}B*\textcolor{red}D = 0}  $$   

    New Assumptions:
    $$ \texttt{\textcolor{red}{A} = 0 or \textcolor{red}{C} = 0} $$
    $$ \texttt{\textcolor{red}{A}*\textcolor{red}D + \textcolor{red}{B}*\textcolor{red}C = \textcolor{red}E} $$
    $$ \texttt{\textcolor{red}B = 0 or \textcolor{red}D = 0}  $$   

    Goal:
    $$\texttt{(\textcolor{red}{A} * \textcolor{red}{D}) = \textcolor{red}{E}  \text{ or } (\textcolor{red}{B} * \textcolor{red}{C}) = \textcolor{red}{E}}$$

\end{frame}

\begin{frame}
    \frametitle{Step 3: Recursively Case on Zero Divisors and Simplify}

    \parbox[t]{5cm}{
    \centering
    Case 1
        $$ \texttt{\textcolor{red}{A} = 0}$$
        $$ \texttt{\textcolor{red}{A}*\textcolor{red}D + \textcolor{red}{B}*\textcolor{red}C = \textcolor{red}E} $$
        $$ \texttt{\textcolor{red}B = 0 or \textcolor{red}D = 0}  $$ }%
    \hspace{1cm}%
    \parbox[t]{5cm}{
    \centering
    Case 2
        $$ \texttt{\textcolor{red}{C} = 0}$$
        $$ \texttt{\textcolor{red}{A}*\textcolor{red}D + \textcolor{red}{B}*\textcolor{red}C = \textcolor{red}E} $$
        $$ \texttt{\textcolor{red}B = 0 or \textcolor{red}D = 0}  $$ }%


    Goal:
    $$\texttt{(\textcolor{red}{A} * \textcolor{red}{D}) = \textcolor{red}{E}  \text{ or } (\textcolor{red}{B} * \textcolor{red}{C}) = \textcolor{red}{E}}$$

\end{frame}

\begin{frame}
    \frametitle{Step 3: Recursively Case on Zero Divisors and Simplify}

    \parbox[t]{5cm}{
    \centering
    Case 1
        $$ \texttt{\textcolor{red}{A} = 0}$$
        $$ \texttt{0*\textcolor{red}D + \textcolor{red}{B}*\textcolor{red}C = \textcolor{red}E} $$
        $$ \texttt{\textcolor{red}B = 0 or \textcolor{red}D = 0}  $$ }%
    \hspace{1cm}%
    \parbox[t]{5cm}{
    \centering
    Case 2
        $$ \texttt{\textcolor{red}{C} = 0}$$
        $$ \texttt{\textcolor{red}{A}*\textcolor{red}D + \textcolor{red}{B}*0 = \textcolor{red}E} $$
        $$ \texttt{\textcolor{red}B = 0 or \textcolor{red}D = 0}  $$ }%


    Goal:
    $$\texttt{(\textcolor{red}{A} * \textcolor{red}{D}) = \textcolor{red}{E}  \text{ or } (\textcolor{red}{B} * \textcolor{red}{C}) = \textcolor{red}{E}}$$

\end{frame}

\begin{frame}

    \frametitle{Step 3: Recursively Case on Zero Divisors and Simplify}

    \parbox[t]{5cm}{
    \centering
    Case 1
        $$ \texttt{\textcolor{red}{A} = 0}$$
        $$ \texttt{\textcolor{red}{B} * \textcolor{red}C = \textcolor{red}E} $$
        $$ \texttt{\textcolor{red}B = 0 or \textcolor{red}D = 0}  $$ }%
    \hspace{1cm}%
    \parbox[t]{5cm}{
    \centering
    Case 2
        $$ \texttt{\textcolor{red}{C} = 0}$$
        $$ \texttt{\textcolor{red}{A} * \textcolor{red}D = \textcolor{red}E} $$
        $$ \texttt{\textcolor{red}B = 0 or \textcolor{red}D = 0}  $$ }%


    Goal:
    $$\texttt{(\textcolor{red}{A} * \textcolor{red}{D}) = \textcolor{red}{E}  \text{ or } (\textcolor{red}{B} * \textcolor{red}{C}) = \textcolor{red}{E}}$$

\end{frame}

\begin{frame}

    \frametitle{Step 3: Recursively Case on Zero Divisors and Simplify}

    \parbox[t]{5cm}{
    \centering
    Case 1
        $$ \texttt{\textcolor{red}{A} = 0}$$
        $$ \texttt{\textcolor{blue}{{B} * C = E}} $$
        $$ \texttt{\textcolor{red}B = 0 or \textcolor{red}D = 0}  $$ }%
    \hspace{1cm}%
    \parbox[t]{5cm}{
    \centering
    Case 2
        $$ \texttt{\textcolor{red}{C} = 0}$$
        $$ \texttt{\textcolor{green}{{A} * D = E}} $$
        $$ \texttt{\textcolor{red}B = 0 or \textcolor{red}D = 0}  $$ }%


    Goal:
    $$\texttt{\textcolor{green}{({A} * {D}) = {E}}  \text{ or } \textcolor{blue}{({B} * {C}) = {E}}}$$
    \begin{center}
        \includegraphics[width=0.2\textwidth]{partypopper.png}        
    \end{center}

\end{frame}

\begin{frame}

    \frametitle{(Sometimes needed) Step 4: Cleanup}

    \begin{itemize}
        \item In real SNARKs there are sums over polynomials, but these can generally be treated as atoms.
        \item SNARKs are based on pairings so the preceding polynomials usually have degree 2.
        \item The preceding steps generally close most goals - by design zeros keep popping up.
        \item ... But usually there are a few critical cases that need to be solved by hand (4 in the case of Groth '16).
    \end{itemize}

\end{frame}

\begin{frame}

    \frametitle{Performance}

    TODO create a table of the performance of different steps for different SNARKs 

\end{frame}

\end{document} 